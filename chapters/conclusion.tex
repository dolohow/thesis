The goal of this work was to implement REST framework for faster development of applications (mainly web) in modern language, in order to drastically decrease development time, and growing complexity and by taking the responsibility of some design decisions. By moving the burden of creating popular modules for setting proper headers, authentication mechanism and so on, development o web application based on this framework is much easier and less time consuming comparing to starting every time from scratch which multiplies by the number of projects created.

Architecture was designed in a such a way that encourages developers to create new plugins for some more specific tasks. Further development of this framework can bring new ones by the means of not breaking the existing functionality which is an important factor for long-lived applications. Current API and model is considered stable. The breakage might be introduced only by a new major versions.

Framework was extensively tested in a production application where it proves itself as a trusted, stable and well performed solution that scales for many users. The applications that was created based on this work is used for exchanging photos between clients and photographers, so pictures can be printed on a paper and send back to the clients. There is a lot of traffic involved during upload of those photos, yet it scales very well. Both mobile (Android) and web application consumes resources exposed by the created framework. By no means it should be very secure so no leakage of private photos will be possible.

Example architecture for testing environment was proposed in order to leverage communication between clients and servers of the main, central server. It can be further modified by adding new servers or providing additional layers depends on needs. The REST prove itself in this scenario as fast and reliable method of communication.

Framework was created in Go for everyone to use free of charge. Anyone can extend it by writing new custom middlewares or contribute to core of it. Most important code fragments were described. This also demonstrates how is easy is to create simple communication mechanism implemented in fairly new language with well designed standard library.

During the work, author of this project struggled a lot with unit testing, but finally he made it through. After the hassle, simple and beautiful framework emerged and all the requirements and expectations have been met.
